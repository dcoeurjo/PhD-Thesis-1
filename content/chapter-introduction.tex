% !TEX root = ../thesis-example.tex
%
\chapter{Introduction et Contexte -- Géométrie digitale et projet digitalSnow}
\label{sec:introduction}

\cleanchapterquote{We have seen that computer programming is an art, because it applies accumulated knowledge to the world, because it requires skill and ingenuity, and especially because it produces objects of beauty.}{Jean-Claude Vandamme}{Ma vie, mon œuvre.}

\setcounter{minitocdepth}{3}
\minitoc

\newpage

introduction
% La géométrie digitale est un domaine relativement jeune

% La création en 1988 de la géométrie arithmétique par JEAN-PIERRE REVEILLÈS
% à Strasbourg provoque une rupture dans la synthèse d’images (une droite, un cercle
% ne sont plus le produit d’un algorithme de tracé mais définis intrinsèquement, sans
% l’être par approximation du continu)

En 2000, durant l'école d'hiver « Digital and Image Geometry » en Allemagne, une liste de « problèmes ouverts » en géométrie et en topologie digitale a été dressé\cite{Klette2000OpenProblems} afin d'orienter les recherches dans le domaines pour les années à venir. Dans cette liste, beaucoup de réponses ont été apportées. On y retrouve par exemple le problème « Algorithme d'estimation et preuves de convergence asymptotique »
